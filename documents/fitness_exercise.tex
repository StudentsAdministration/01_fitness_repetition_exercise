\hypertarget{fitness-repetition-exercise}{%
\section{\#1: Fitness Repetition
Exercise}\label{fitness-repetition-exercise}}

\hypertarget{added-teachers-materials}{%
\subsection{Added teachers materials}\label{added-teachers-materials}}

\begin{itemize}
\tightlist
\item
  .docx, .pdf, .epub versions
\item
  \href{/documents}{Documents folder}
\end{itemize}

In this exercise you are going to make a program that can show lists of
Employees and members of a Fitness Club. \#\# Employees There are 2
types of Employees: * Administration personnel * Instructors

The administration personnel are employed full time (37 hours a week),
and all have a fixed salary of 23000 a month, and have 5 weeks of
vacation a year.

Instructors a hired on an hourly basis, and are payed 199 per hour. They
don't have any paid vacation.

\hypertarget{members}{%
\subsection{Members}\label{members}}

The members can have 2 types of membership * Basic * Full The Basic
membership costs 199.- a month, and the Full membership 299,-

You should make a program that consists of the following Classes and
uses inheritance in the specified way:

\begin{figure}
\centering
\includegraphics{img/classdiagram.png}
\caption{Fitness Class Diagram}
\end{figure}

Then make a file containing a main method and call the file
FitnessMain.java. In this file you should write some code that prints
out something like this in your console. (the data (names, cpr etc.) are
up to you to decide, they don't have to be the same as here).

\begin{figure}
\centering
\includegraphics{img/consoleoutput1.png}
\caption{Console Output}
\end{figure}

\hypertarget{read-data-from-a-text-file}{%
\subsection{Read data from a text
file}\label{read-data-from-a-text-file}}

Now make a text file named persons.txt. This file should contain the
name and cpr of all employees and all members. Every person should be on
a separate line and there should be a blank space between the name and
the cpr number.

Make a new file called fileHandling.java and in this file should be the
code for reading from the persons.txt file and adding it to an ArrayList
of Person. You will have to use a Scanner Object to read from the file.
Make one method that reads from the text file and adds it to the
ArrayList (this method could be private). and make a get method that
returns an ArrayList of Person and call that method in your main method.
Then print out the content in the console. The console should look
something like this:

\begin{figure}
\centering
\includegraphics{img/consoleoutput2.png}
\caption{Console Output}
\end{figure}

If you made it this far you should be happy !!

If you feel like it you could do the same for Employees only and Members
Only. Keep it ``simple'' and create a unique text file for each so you
will end up with 3 text files: persons.txt, employees.txt and
members.txt

\hypertarget{xtra}{%
\subsection{Xtra}\label{xtra}}

Make an adjustment to the program so the user can search for a member or
employee by typing the name or cpr number of a person in the console.
Then show this person's data in the console. You have to use the
ArrayList method ``contains'' to search an ArrayList (p.~682 in the
book).

\hypertarget{xtra-2}{%
\subsection{Xtra 2}\label{xtra-2}}

Make your program able to add a new user to the text files through input
from the console.

\_

© clbo@kea.dk

\_
